


\section{L\'ogica Proposicional}

\subsection{Sintaxis de la L\'ogica Proposicional}

Usaremos variables proposicionales para indicar proposiciones {\em completas} e {\em indivisibles}.
En general llamaremos $P$ al conjunto de variables proposicionales, y, por ejemplo,
$p$, $q$, $r$, socrates\_es\_hombre, a las variables mismas.

\begin{definicion}
Sea $P$ un conjunto de variables proposicionales.
El conjunto de todas las {\em f\'ormulas} de l\'ogica proposicional sobre $P$,
denotado por $L(P)$, se define inductivamente por:
\begin{itemize}
\item Si $p\in P$, entonces $p$ es una f\'ormula en $L(P)$.
\item Si $\varphi\in L(P)$, entonces $(\neg \varphi)$ es una f\'ormula en $L(P)$
\item Si $\varphi,\psi\in L(P)$, entonces $(\varphi \wedge \psi)$,
$(\varphi \vee \psi)$, $(\varphi \to \psi)$, y $(\varphi \ssi \psi)$ son f\'ormulas en $L(P)$
\end{itemize}
\end{definicion}

\subsection{Sem\'antica de la L\'ogica Proposicional}

>Cu\'ando una f\'ormula es verdadera? depende del {\em mundo} en el que la estamos interpretando.
Un mundo particular le dar\'a una interpretaci\'on a cada variable proposicional, le dar\'a un
valor {\em verdadero} o {\em falso} a cada variable.

Una {\em valuaci\'on} (o {\em asignaci\'on de verdad}) para las variables de
$P$ es una funci\'on, $\sigma:P\to \{0,1\}$.

\begin{definicion}
Sea $P$ un conjunto de variables proposicionales y $\sigma$ una asignaci\'on de verdad para $P$.
Dada una f\'ormula $\varphi$ en $L(P)$, se definimos la funci\'on $\hat\sigma:L(P)\to \{0,1\}$, 
por inducci\'on como sigue:
\begin{itemize}
\item Si $p\in P$ entonces $\hat\sigma(p)=\sigma(p)$.
\item \[\hat\sigma(\, (\neg \varphi)\, )= 
      \left\{ 
      \begin{array}{cl}
      1 & \text{si }\hat\sigma(\varphi)=0, \\
      0 & \text{si }\hat\sigma(\varphi)=1. \\
      \end{array}
      \right.
\]
\item
\[\hat\sigma(\, (\varphi \vee \psi)\, )= 
      \left\{ 
      \begin{array}{cl}
      1 & \text{si }\hat\sigma(\varphi)=1\text{ o }\hat\sigma(\psi)=1, \\
      0 & \text{si }\hat\sigma(\varphi)=0\text{ y }\hat\sigma(\psi)=0. \\
      \end{array}
      \right.
\]

\[\hat\sigma(\, (\varphi \wedge \psi)\, )= 
      \left\{ 
      \begin{array}{cl}
      1 & \text{si }\hat\sigma(\varphi)=1\text{ y }\hat\sigma(\psi)=1, \\
      0 & \text{si }\hat\sigma(\varphi)=0\text{ o }\hat\sigma(\psi)=0. \\
      \end{array}
      \right.
\]

\[\hat\sigma(\, (\varphi \to \psi)\, )= 
      \left\{ 
      \begin{array}{cl}
      1 & \text{si }\hat\sigma(\varphi)=0\text{ o }\hat\sigma(\psi)=1, \\
      0 & \text{si }\hat\sigma(\varphi)=1\text{ y }\hat\sigma(\psi)=0. \\
      \end{array}
      \right.
\]

\[\hat\sigma(\, (\varphi \ssi \psi)\, )= 
      \left\{ 
      \begin{array}{cl}
      1 & \text{si }\hat\sigma(\varphi)=\hat\sigma(\psi), \\
      0 & \text{si }\hat\sigma(\varphi)\neq\hat\sigma(\psi). \\
      \end{array}
      \right.
\]
\end{itemize}
Diremos que $\hat\sigma(\varphi)$ es la {\em evaluaci\'on} de $\varphi$ dada la 
asignaci\'on $\sigma$.
\end{definicion}

De ahora en adelante llamaremos simplemente $\sigma$ a $\hat\sigma$, y denotaremos a la
a la evaluaci\'on de la f\'ormula $\varphi$ dada la asignaci\'on $\sigma$, simplmente como $\sigma(\varphi)$.

\begin{definicion}
Las f\'ormulas $\varphi,\psi\in L(P)$ son {\em l\'ogicamente equivalentes}, si para toda asignaci\'on de verdad $\sigma$ se tiene
que $\sigma(\varphi)=\sigma(\psi)$. 
Denotaremos por $\equiv$ la equivalencia l\'ogica, as\'{i}, si $\varphi$ y $\psi$ son l\'ogicamente equivalentes, escribiremos
$\varphi\equiv \psi$.
\end{definicion}

\begin{ejemplo}
Demostraremos que las f\'ormulas $(p\to q)$ y $((\neg p) \vee q)$ son l\'ogicamente equivalentes.
Dado que ambas f\'ormulas tienen s\'olo dos variables proposicionales, la cantidad de valuaciones $\sigma:\{p,q\}\to \{0,1\}$
son 4. Podemos probarlas todas en una tabla como la que sigue:

\begin{center}
\begin{tabular}{lcccc} %\cline{2-5} 
& ~$p$~ & ~$q$~ & ~$(p\to q)$~ & ~$((\neg p) \vee q)$~ \\ \cline{2-5} 
$\sigma_1$: & 0 & 0 & 1 & 1\\ %\cline{2-5}
$\sigma_2$: & 0 & 1 & 1 & 1\\ %\cline{2-5}
$\sigma_3$: & 1 & 0 & 0 & 0\\ %\cline{2-5}
$\sigma_4$: & 1 & 1 & 1 & 1\\ %\cline{2-5}
\end{tabular}
\end{center}

Cada fila de la anterior tabla corresponde a una asignaci\'on de verdad diferente, las dos primeras columnas
corresponden a las asignaciones a las variables, y las dos siguientes al valor de verdad asignado a cada f\'ormula.
En este caso, ambas f\'ormulas tienen exactamente el mismo valor de verdad para cada posible asignaci\'on por lo tanto
son equivalentes. Concluimos entonces que $(p\to q)\equiv ((\neg p)\vee q)$.
\end{ejemplo}

El anterior es un ejemplo del uso de {\em Tablas de Verdad}, que son tablas en las que las filas representan
todas las posibles valuaciones para un conjunto de variables proposicionales, 
y las columnas, etiquetadas con una f\'ormula particular,
representan los distintos valores de verdad de la f\'ormula en cada valuaci\'on.
Note que dos f\'ormulas son l\'ogicamente equivalentes, si y solo si, su respectivas columnas en una tabla de verdad
contienen exactamente la misma secuencia de valores.
Las tablas tambi\'en nos permiten establecer algunas propiedades de conteo.
Suponga que $P=\{p_1,p_2,\ldots, p_n\}$, >cu\'antas f\'ormulas no equivalentes hay en $L(P)$? 
Calcule este n\'umero y contrastelo con la cantidad de f\'ormulas distintas en $L(P)$.

Se puede demostrar que el reemplazo de sub-f\'ormulas equivalentes en una f\'ormula, no
altera el valor de verdad de la f\'ormula original.
M\'as formalmente, sea $\varphi$ una f\'ormula que contiene a $\psi$ como sub-f\'ormula, y sea
$\psi'$ una f\'ormula tal que $\psi\equiv \psi'$. 
Si $\varphi'$ es la f\'ormula obtenida de $\varphi$ reemplazando $\psi$ por $\psi'$, entonces $\varphi\equiv \varphi'$.
(Haga la demostraci\'on de esta \'ultima propiedad por inducci\'on estructural. Para
esto primero deber\'a definir inductivamente el concepto de sub-f\'ormula y formalizar
lo que significa reemplazar una sub-f\'ormula por otra).

Las siguientes son algunas equivalencias \'utiles (demuestre que se cumplen):
\begin{enumerate}
\item $(\varphi \vee \psi) \equiv (\psi \vee \varphi)$
\item $(\varphi \wedge \psi) \equiv (\psi \wedge \varphi)$
\item $(\varphi \vee (\psi \vee \chi)) \equiv ((\varphi \vee \psi) \vee \chi)$
\item $(\varphi \wedge (\psi \wedge \chi)) \equiv ((\varphi \wedge \psi) \wedge \chi)$
\item $(\varphi \vee (\psi \wedge \chi)) \equiv ((\varphi \vee \psi) \wedge (\varphi \vee \chi))$
\item $(\varphi \wedge (\psi \vee \chi)) \equiv ((\varphi \wedge \psi) \vee (\varphi \wedge \chi))$
\item $(\neg (\varphi \wedge \psi))\equiv ((\neg \varphi) \vee (\neg \psi))$
\item $(\neg (\varphi \vee \psi))\equiv ((\neg \varphi) \wedge (\neg \psi))$
\item $(\neg(\neg \varphi))\equiv \varphi$
\end{enumerate}

Las reglas 3 y 4, nos permiten evitar par\'entesis cuando consideramos secuencias de f\'ormulas operadas
usando $\vee$ y $\wedge$, respectivamente. 
De ahora en adelante escribiremos simplemente $\varphi_1 \vee \varphi_2 \vee \varphi_3$ (sin usar
par\'entesis de asociaci\'on).
Adicionalmente escribiremos 
\[
\bigvee_{i=1}^{n}\varphi_i = \varphi_1 \vee \varphi_2 \vee \cdots \vee \varphi_n.
\] Similarmente lo haremos con $\wedge$.

\begin{definicion}
Una f\'ormula $\varphi\in L(P)$ es:
\begin{itemize}
\item {\em Tautolog\'{i}a} si para toda valuaci\'on $\sigma$ se tiene que $\sigma(\varphi)=1$,
\item {\em Satisfacible} si existe una valuaci\'on $\sigma$ tal que $\sigma(\varphi)=1$,
\item {\em Contradicci\'on} si no es satisfacible (o sea, para toda valuaci\'on $\sigma$ se tiene que $\sigma(\varphi)=0$). 
\end{itemize}
\end{definicion}

Considere una f\'ormula $\varphi$ en $L(P)$ con $P=\{p,q,r\}$, donde lo \'unico que conocemos de $\varphi$ es
que cumple la siguiente tabla de verdad:

\begin{center}
\begin{tabular}{ccc|cc} %\cline{2-5} 
~$p$~ & ~$q$~ & ~~$r$~~ & ~$\varphi$~ \\ \hline 
0 & 0 & 0 & 1\\ %\cline{2-5}
0 & 0 & 1 & 0\\ %\cline{2-5}
0 & 1 & 0 & 0\\ %\cline{2-5}
0 & 1 & 1 & 1\\ %\cline{2-5}
1 & 0 & 0 & 1\\ %\cline{2-5}
1 & 0 & 1 & 0\\ %\cline{2-5}
1 & 1 & 0 & 0\\ %\cline{2-5}
1 & 1 & 1 & 1\\ %\cline{2-5}
\end{tabular}
\end{center}

>Podemos usando s\'olo esta informaci\'on, construir efectivamente una f\'ormula que sea equivalente a $\varphi$?
>Qu\'e operadores necesitamos para hacerlo?
La siguiente f\'ormula muestra una respuesta positiva a la primera pregunta:
\[
((\neg p) \wedge (\neg q) \wedge (\neg r))\; \vee\; 
((\neg p) \wedge q \wedge r)\; \vee\;
(p \wedge (\neg q) \wedge (\neg r))\; \vee\;
(p \wedge q \wedge r).
\]
La idea de la anterior f\'ormula es imitar la manera en que la valuaci\'on hace verdadera a $\varphi$.
De hecho tiene exactamente la misma tabla de verdad que $\varphi$. 
Podemos generalizar el anterior argumento para cualquier f\'ormula dada su tabla de verdad de la
siguiente manera.
Considere una f\'ormula $\varphi$ en donde ocurren $n$ variables proposicionales $p_1,p_2,\ldots,p_n$, y sean
$\sigma_1,\sigma_2,\ldots,\sigma_{2^n}$, una enumeraci\'on de todas las posibles valuaciones
para las variables en $\varphi$.
Para cada $\sigma_j$ con $j=1,\ldots,2^n$ %tal que $\sigma_j(\varphi)=1$ 
considere la siguiente f\'ormula $\varphi_j$
\[
\varphi_j=\big(\underset{\substack{i=1\ldots n\\ \sigma_j(p_i)=1}}{\bigwedge}p_i\big) 
\wedge \big( \underset{\substack{i=1\ldots n\\ \sigma_j(p_i)=0}}{\bigwedge}(\neg p_i) \big).
\]
Note que $\varphi_j$ representa a la fila $j$ de la tabla de verdad para $\varphi$.
Por ejemplo, en el caso de la f\'ormula y la tabla de verdad de m\'as arriba, suponiendo que las 
filas se numeran desde arriba abajo, tenemos que 
$\varphi_5=(p \wedge (\neg q) \wedge (\neg r))$.
Lo \'unico que falta ahora es hacer la disyunci\'on de todas las f\'ormulas $\varphi_j$ para $j$ entre $1$ y $2^n$
tal que $\sigma_j(\varphi)=1$. Finalmente obtenemos la f\'ormula
\[
\bigvee_{\substack{j=1\ldots 2^n\\ \sigma_j(\varphi)=1}}\varphi_j = 
\bigvee_{\substack{j=1\ldots 2^n\\ \sigma_j(\varphi)=1}}\bigg( 
\big(\underset{\substack{i=1\ldots n\\ \sigma_j(p_i)=1}}{\bigwedge}p_i\big) 
\wedge \big( \underset{\substack{i=1\ldots n\\ \sigma_j(p_i)=0}}{\bigwedge}(\neg p_i) \big) \bigg).
\]
Se puede demostrar (h\'agalo de ejercicio) que esta \'ultima f\'ormula es l\'ogicamente equivalente
a $\varphi$.
El \'unico detalle que nos falta es que, si $\varphi$ es una contradicci\'on, la f\'ormula de m\'as arriba
es ``vac\'{i}a'' ya que para toda valuaci\'on $\sigma_j$ se tendr\'{i}a que $\sigma_j(\varphi)=0$.
Pero en este \'ultimo caso, podr\'{i}amos expresar $\varphi$ como $(p\wedge (\neg p))$.

Hemos demostrado entonces que cualquier tabla de verdad puede ser representada con una f\'ormula,
y m\'as a\'un, con una f\'ormula que s\'olo usa los conectivos l\'ogicos $\neg$, $\vee$ y $\wedge$.
Esto motiva la siguiente definici\'on.

\begin{definicion}
Un conjunto de operadores l\'ogico $C$ se dice {\em funcionalmente completo},
si toda f\'ormula en $L(P)$ es l\'ogicamente equivalente a una f\'ormula que usa s\'olo operadores
en $C$.
\end{definicion}

Ya demostramos que $\{\neg,\vee,\wedge\}$ es funcionalmente completo. 
Como ejercicio, demuestre que $\{\neg, \vee\}$ es tambi\'en funcionalmente
completo (note que para hacer esta demostraci\'on, basta con encontrar una forma de expresar
la conjunci\'on $(\varphi \wedge \psi)$ usando s\'olo $\neg$ y $\vee$, de todas maneras se necesita un
argumento inductivo).

\subsection{Formas Normales}

Un {\em literal} es una variable proposicional o la negaci\'on de una variable proposicional, por ejemplo,
$p$ y $(\neg r)$ son ambos literales. 
De ahora en adelante supondremos que $\neg$ tienen {\em presedencia} sobre $\vee$ y $\wedge$, y por lo tanto
podremos escribir una f\'ormula como $((\neg p) \vee q) \wedge (\neg r)$, simplemente como $(\neg p \vee q) \wedge \neg r$.
Entonces, por ejemplo, $\neg p$, $q$ y $\neg r$ son literales.

\begin{definicion}
Una f\'ormula $\varphi$ est\'a en {\em Forma Normal Disyuntiva} (FND), si es una disyunci\'on de conjunciones de literales,
o sea, si es de la forma
\[
B_1 \; \vee \; B_2\;  \vee\; \cdots\; \vee\; B_k
\]
donde cada $B_i$ es una conjunci\'on de literales, $B_i=(\ l_{i1} \wedge l_{i2} \wedge \cdots \wedge l_{ik_i}\ )$.
Una f\'ormula $\psi$ est\'a en {\em Froma Normal Conjuntiva} (FNC), si es una conjunci\'on de disyunciones de literales,
o sea, si es de la forma
\[
C_1 \; \wedge \; C_2\;  \wedge \; \cdots\; \wedge \; C_k
\]
donde cada $C_i$ es una disyunci\'on de literales, $C_i=(\ l_{i1} \vee l_{i2} \vee \cdots \vee l_{ik_i}\ )$.
A una disyunci\'on de literales se le llama {\em cl\'ausula}, por ejemplo, cada una de las $C_i$ anteriores
es una cl\'ausula.
Entonces, una f\'ormula est\'a en FNC, si es una conjunci\'on de cl\'ausulas.
\end{definicion}

\begin{teorema}
\begin{enumerate}
\item Toda f\'ormula en $L(P)$ es l\'ogicamente equivalente a una f\'ormula en FND
\item Toda f\'ormula en $L(P)$ es l\'ogicamente equivalente a una f\'ormula en FNC
\end{enumerate}
\end{teorema}

\begin{demostracion}
\begin{enumerate}
\item Ya lo hicimos cuando mostramos como representar una tabla de verdad con una f\'ormula.
\item Ejercicio. 
\end{enumerate}
\end{demostracion}

\subsection{Consecuencia L\'ogica}

Sea $P$ un conjunto de variables proposicionales.
Dado un conjunto de f\'ormulas $\Sigma$ en $L(P)$ y una valuaci\'on $\sigma$ para las variables en $P$,
diremos que $\sigma$ satisface $\Sigma$ si para toda f\'ormula $\varphi\in \Sigma$ se tiene que $\sigma(\varphi)=1$.
En este caso escribimos $\sigma(\Sigma)=1$.

\begin{definicion}
Sea $\Sigma$ un conjunto de f\'ormulas en $L(P)$ y $\psi$ una f\'ormula en $L(P)$, diremos que $\psi$ es
{\em consecuencia l\'ogica} de $\Sigma$, si para toda valuaci\'on $\sigma$ tal que $\sigma(\Sigma)=1$,
se tiene que $\sigma(\psi)=1$.
En este caso escribiremos $\Sigma \models \psi$.
\end{definicion} 

\begin{ejemplo}
\begin{itemize}
\item $\{p,\ p\to q\}\models q$ ({\em Modus Ponens})
\item $\{\neg q,\ p\to q\}\models \neg p$ ({\em Modus Tollens})
\item $\{p\vee q\vee r,\ p\to s,\ q\to s,\ r\to s\}\models s$ (Demostraci\'on por partes)
\item $\{p\vee q,\ \neg q\vee r\} \models p\vee r$ (Resoluci\'on)
\end{itemize}
\end{ejemplo}

\begin{definicion}
Un conjunto de f\'ormulas $\Sigma$ es {\em inconsistente} si no existe una valuaci\'on $\sigma$ tal que 
$\sigma(\Sigma)=1$.
El conjunto $\Sigma$ es {\em satisfacible}, si existe una valuaci\'on $\sigma$ tal que $\sigma(\Sigma)=1$. 
\end{definicion}

\begin{teorema}
La f\'ormula $\varphi$ es consecuencia l\'ogica de $\Sigma$, si y solo si, el conjunto $\Sigma \cup \{\neg \varphi\}$
es inconsistente.
\begin{demostracion}
$(\Rightarrow)$ Suponga que $\Sigma\models \varphi$, demostraremos que $\Sigma \cup \{\neg \varphi\}$ es inconsistente.
Lo haremos por contradicci\'on. Entonces, supongamos que $\Sigma \cup \{\neg \varphi\}$ es consistente.
Esto implica que existe una valuaci\'on $\sigma$ tal que $\sigma(\Sigma\cup\{\neg \varphi\})=1$, lo que implica que 
$\sigma(\Sigma)=1$ y $\sigma(\neg \varphi)=1$, y por lo tanto $\sigma(\Sigma)=1$ y 
$\sigma(\varphi)=0$, lo que contradice el hecho de que $\Sigma\models \varphi$.

$(\Leftarrow)$ Supongamos que $\Sigma\cup \{\neg \varphi\}$ es inconsistente, demostraremos que $\Sigma \models \varphi$.
Sea $\sigma$ una valuaci\'on tal que $\sigma(\Sigma)=1$, debemos demostrar que $\sigma(\varphi)=1$.
Dado que $\Sigma\cup \{\neg \varphi\}$ es inconsistente, y $\sigma$ es una valuaci\'on tal que $\sigma(\Sigma)=1$, 
necesariamente se tiene que $\sigma(\neg \varphi)=0$, de lo que concluimos que $\sigma(\varphi)=1$.
Hemos demostrado que, si $\sigma$ es tal que $\sigma(\Sigma)=1$, 
entonces $\sigma(\varphi)=1$, lo que implica que $\Sigma\models \varphi$.
\end{demostracion}
\end{teorema}

Entonces para chequear que $\Sigma\models \varphi$, basta con chequear que $\Sigma\cup \{\neg \varphi\}$ es inconsistente.
>C\'omo chequeamos que un conjunto de f\'ormulas es inconsistente?


La primera observaci\'on es que podemos extender la idea de equivalencia a conjuntos de f\'ormulas.
Dos conjuntos $\Sigma_1$ y $\Sigma_2$ son l\'ogicamente equivalentes (y escribimos $\Sigma_1\equiv \Sigma_2$),
si para toda valuaci\'on $\sigma$ se tiene que $\sigma(\Sigma_1)=\sigma(\Sigma_2)$.
Similarmente diremos que $\Sigma$ es l\'ogicamente equivalente a la f\'ormula $\varphi$, si
$\Sigma\equiv \{\varphi\}$.

La segunda observaci\'on es que, todo conjunto $\Sigma$ es equivalente a la conjunci\'on de sus f\'ormulas
\[
\Sigma \equiv \bigwedge_{\varphi \in \Sigma}\varphi.
\]
Adem\'as sabemos que toda f\'ormula es equivalente a una en FNC de la forma $C_1\wedge C_2\wedge \cdots \wedge C_n$,
donde cada $C_i$ es un cl\'ausula.
Por otra parte, una f\'ormula en FNC es l\'ogicamente equivalente al conjunto de sus cl\'ausulas.
De toda esta discusi\'on obtenemos que todo conjunto de f\'ormulas es l\'ogicamente equivalente a un
conjunto de cl\'ausulas.
Para obtener el conjunto de cl\'ausulas correspondiente, primero llevamos la conjunci\'on de
f\'ormulas del primer conjunto a una f\'ormula equivalente en FNC, y luego creamos el conjunto
de todas las cl\'ausulas obtenidas.

\begin{ejemplo}
$\{p,\ q\to (p\to r),\ \neg(q\to r)\} \equiv \{p,\ \neg q\vee \neg p\vee r,\ q,\ \neg r\}$
\end{ejemplo}

Queremos un m\'etodo para chequear cuando un conjunto de cl\'ausulas $\Sigma$ es inconsistente.
Sea $\varphi$ una f\'ormula que representa una contradicci\'on (por ejemplo $p\wedge \neg p$).
Vamos a introducir un nuevo s\'{i}mbolo $\square$, para representar una f\'ormula gen\'erica que es contradicci\'on.
O sea, $\square$ es una f\'ormula tal que para toda valuaci\'on se tiene que $\sigma(\square)=0$.
Llamamos a $\square$ la {\em cl\'ausula vac\'{i}a}.
No es dificil demostrar (h\'agalo de ejercicio) que $\Sigma$ es inconsistente si y solo si
$\Sigma\models \square$.

Lo que veremos es un m\'etodo, llamado m\'etodo de resoluci\'on, que usa un sistema de reglas 
para, dado un conjunto de cl\'ausulas $\Sigma$ determinar si $\Sigma\models \square$ y por lo tanto, 
determinar si $\Sigma$ es inconsistente. Primero introduciremos un poco de notaci\'on.
Sea $\ell$ un literal, si $\ell$ es igual a una variable proposicional $p$, entonces $\bar{\ell}$
representa a $\neg p$. Similarmente, si $\ell=\neg p$ entonces $\bar{\ell}=p$.
La regla que est\'a en el coraz\'on del m\'etodo, se llama {\em Regla de Resoluci\'on} y 
tiene la siguiente forma:
\[
\begin{array}{c}
C_1 \vee \ell \vee C_2 \\
C_3 \vee \bar{\ell} \vee C_4 \\ \hline
C_1 \vee C_2 \vee C_3 \vee C_4 
\end{array}
\]
con $C_1$, $C_2$, $C_3$, $C_4$ cl\'ausulas y $\ell$ un literal.
Esta es una regla {\bf sint\'actica}, que genera un nuevo objeto dados dos objetos
anteriores.
La idea es que si tengo dos cl\'ausulas tales que, en una aparece un literal $\ell$ y en la otra aparece la
negaci\'on $\bar{\ell}$ de ese literal, entonces genero una nueva cl\'ausula como la disyunci\'on de ambas
sin considerar $\ell$ ni $\bar{\ell}$.
Sem\'anticamente, esta regla es {\em correcta}, de hecho es f\'acil ver que (demu\'estrelo)
\[
\{ C_1 \vee \ell \vee C_2,\ 
C_3 \vee \bar{\ell} \vee C_4\} \models 
C_1 \vee C_2 \vee C_3 \vee C_4 
\]
Algunos casos particulares de la regla de resoluci\'on son los siguientes:
\[
\begin{array}{c}
C_1 \vee \ell \vee C_2 \\
\bar{\ell} \\ \hline
C_1 \vee C_2 
\end{array}
\;\;\;\;\;\;\;\;
\;\;\;\;\;\;\;\;
\begin{array}{c}
\ell \\
\bar{\ell} \\ \hline
\square 
\end{array}
\]
\begin{ejemplo}
Un ejemplo de aplicaci\'on de la regla de resoluci\'on
\[
\begin{array}{c}
\neg p \vee q \\
\neg q \vee r \vee s \\ \hline
\neg p \vee r \vee s
\end{array}
\]
Entonces podemos concluir que $\{\neg p \vee q,\ \neg q \vee r \vee s\}\models \neg p \vee r \vee s$.
\end{ejemplo}

Adicionalmente necesitamos la regla de factorizaci\'on, que esencialmente dice
que si un literal se repite en una cl\'ausula, entonces se puede eliminar una de las repeticiones:
\[
\begin{array}{c}
C_1 \vee \ell \vee C_2 \vee \ell \vee C_3 \\ \hline
C_1 \vee \ell \vee C_2 \vee C_3
\end{array}
\]
\begin{definicion}
Dado un conjunto $\Sigma$ de cl\'ausulas, una demostraci\'on por resoluci\'on de que $\Sigma$
es inconsistente es una secuencia de cl\'ausulas $C_1,C_2,\ldots,C_n$ tal que
$C_n=\square$ y para cada $i=1,\ldots, n$ se tiene que:
\begin{itemize}
\item $C_i\in \Sigma$, o
\item $C_i$ se obtiene de dos cl\'ausulas anteriores en la secuencia usando la regla de resoluci\'on, o
\item $C_i$ se obtiene de una cl\'asula anterior en la secuencia usando la regla de factorizaci\'on.
\end{itemize}
Si existe tal demostraci\'on, escribimos $\Sigma\vdash \square$.
\end{definicion}

\begin{ejemplo}
La siguiente es una demostraci\'on por resoluci\'on de que el conjunto 
\[
\Sigma=\{p\vee q\vee r,\ \neg p \vee s,\ \neg q\vee s,\ \neg r\vee s, \neg s\}
\]
es inconsistente.
\[
\begin{array}{cll}
(1) & p\vee q\vee r & \text{est\'a en }\Sigma \\
(2) & \neg p\vee s & \text{est\'a en }\Sigma\\
(3) & s\vee q\vee r & \text{resoluci\'on (1) y (2)} \\
(4) & \neg q \vee s & \text{est\'a en }\Sigma \\
(5) & s \vee s \vee r & \text{resoluci\'on (3) y (4)}\\
(6) & s \vee r & \text{factorizaci\'on (5)} \\
(7) & \neg r \vee s & \text{est\'a en }\Sigma \\
(8) & s \vee s & \text{resoluci\'on (6) y (7)} \\
(9) & s & \text{factorizaci\'on (8)} \\
(10) & \neg s & \text{est\'a en }\Sigma \\
(11) & \square & \text{resoluci\'on (9) y (10)}
\end{array}
\]
\end{ejemplo}

\begin{teorema}
Dado un conjunto de cl\'ausulas $\Sigma$ se tiene que:
\begin{description}
\item[(Correctitud)] Si $\Sigma\vdash \square$ entonces $\Sigma$ es inconsistente.
\item[(Completitud)] Si $\Sigma$ es inconsistente entonces $\Sigma\vdash \square$. 
\end{description}
\end{teorema}

\begin{ejemplo}
Usaremos resoluci\'on para demostrar que 
$\{p,\ q\to(p\to r)\} \models q\to r$.
Primero, sabemos que $\{p,\ q\to(p\to r)\} \models q\to r$, si y s\'olo si el conjunto 
$\{p,\ q\to(p\to r),\ \neg(q\to r)\}$
es inconsistente.
Conviertiendo cada f\'ormula en FNC, notamos que este \'ultimo conjunto de f\'ormulas es l\'ogicamente equivalente al conjunto de cl\'ausulas
\[
\Sigma= \{p,\ \neg q\vee \neg p\vee r,\ q,\ \neg r\}.
\]
Basta entonces demostrar que $\Sigma$ es inconsistente, o equivalentemente que $\Sigma\vdash \square$.
La siguiente es una demostraci\'on de esto \'ultimo:
\[
\begin{array}{cll}
(1) & p & \text{est\'a en }\Sigma \\
(2) & \neg q\vee \neg p\vee r & \text{est\'a en }\Sigma\\
(3) & \neg q \vee r & \text{resoluci\'on (1) y (2)} \\
(4) & q & \text{est\'a en }\Sigma \\
(5) & r & \text{resoluci\'on (3) y (4)}\\
(6) & \neg r & \text{est\'a en }\Sigma \\
(7) & \square & \text{resoluci\'on (5) y (6).}
\end{array}
\]
\end{ejemplo}
