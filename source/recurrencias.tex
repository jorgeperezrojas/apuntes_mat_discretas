\section{Relaciones de Recurrencia}
En un ejemplo de la sección anterior se introdujo la sucesión de Fibonacci, con la siguiente definición recursiva:
\[
\begin{array}{rcl}
F_0 & = & 0 \\
F_1 & = & 1 \\
F_n & = & F_{n-1} + F_{n-2} \;\;\;\forall n\geq 2
\end{array}
\]
La definición nos dice que el valor del $n$--ésimo término de la sucesión está dado por la suma de los dos valores anteriores.
¿Es posible encontrar una función que nos entregue de manera no recursiva el valor de $F_n$?
¿Es posible al menos estimar el valor de $F_n$?
A la última de estas preguntas, ya le dimos respuesta, de hecho demostramos que $F_n$ es siempre menor que $2^n$, incluso usando una técnica similar se puede demostrar que $F_n$ es mayor que $(3/2)^n$ para todo $n\geq 5$, con esto tenemos que si $n\geq 5$
\[
\left(\frac{3}{2}\right)^n<F_n<2^n
\]
A pesar de que estas desigualdades nos dan una idea del valor de $F_n$ (muy muy grande, incluso para valores pequeños de $n$) aún no tenemos una función que nos dé un valor exacto.
En esta sección estudiaremos ecuaciones recursivas y cómo estas aparecen en definiciones de conceptos cotidianos tanto en computación, como en matemáticas, y las maneras de obtener estimaciones de sus valores.

\begin{definicion}
Sea $S$ una secuencia de valores $s_0,s_1,s_2,\ldots$. Una \emph{relación de recurrencia} para $S$ es una ecuación que define el valor de cada uno de los $s_n$ en función de algunos de sus predecesores $s_{n-1},s_{n-2},\ldots,s_1,s_0$, salvo por un conjunto de términos \emph{base} que se definen explícitamente.
\end{definicion}

La definición de relación de recurrencia tiene un ``sabor inductivo'', de hecho, inducción será una de las principales armas que usaremos para atacar nuestro problema.
Iniciaremos el estudio con un par de ejemplos de problemas recurrentes.

\begin{ejemplo}[Las Torres de Hanoi.]
Existen tres mástiles uno de los cuales tiene discos todos de distintos radios en él (los discos tienen un agujero central por donde se insertan en el mástil).
Estos discos están dispuestos de manera decreciente de sus radios, el disco de mayor radio está al fondo, el de menor radio al tope.
El problema es el siguiente: Se quieren pasar todos los discos desde el mástil inicial a uno de los otros, moviendo un disco a la vez y de manera tal que durante todo el proceso un disco particular nunca esté sobre un disco de radio mayor.
La figura~\ref{hanoi} muestra la posición inicial.

\begin{figure}[h!]
\caption{Las Torres de Hanoi.}
\label{hanoi}
\end{figure}

Queremos encontrar cuál es el mínimo número movimientos de discos que debemos hacer para pasarlos todos desde el primer mástil a algún otro.
Llamemos a este número $T_n$, donde $n$ representa la cantidad de discos que se encuentran inicialmente puestos en el mástil.
Primero observamos casos pequeños, es claro que $T_0=0$, $T_1=1$ y que $T_2=3$.
>Cómo lo hacemos para $n=3$?
La idea ``ganadora'' es mover los primeros dos discos a uno de los mástiles, luego mover el tercer disco al mástil aún vacío y finalmente mover los dos discos sobre el tercero.
Esto nos da la pista para mover $n$ discos: primero movemos los $n-1$ discos más pequeños a uno de los mástiles, luego movemos el disco más grande, y finalmente (nuevamente) movemos los $n-1$ discos al mástil sobre el disco de mayor radio.
En el primer paso (mover los $n-1$ discos) tuvimos que realizar $T_{n-1}$ movimientos en total, el segundo paso (el disco más grande) significó un sólo movimiento, y el tercer paso (nuevamente $n-1$ discos) significó $T_{n-1}$ movimientos, en total para mover los $n$ discos de un mástil a otro necesitamos $2T_{n-1}+1$ movimientos.
Tenemos entonces la siguiente relación de recurrencia:
\[
\begin{array}{rcl}
T_0 & = & 0 \\
T_n & = & 2T_{n-1} + 1 \;\;\;\forall n\geq 1
\end{array}
\]
Es válido preguntarse si $T_n$ definido por la recurrencia anterior es efectivamente la menor cantidad de movimientos necesarios para resolver el problema cuando hay $n$ discos.
La respuesta es sí, de hecho puede demostrarse por inducción en $n$ que cualquier otra solución necesitará al menos $T_n$ movimientos.
\end{ejemplo}

\begin{ejemplo}
>Cuál es la máxima cantidad de trozos que se pueden obtener haciendo $n$ cortes rectos a una torta?
O más académicamente, >Cuál es la máxima cantidad de regiones definidas por $n$ rectas que se pueden obtener en el plano?
\end{ejemplo}



